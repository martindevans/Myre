\documentclass{article}

\usepackage[utf8]{inputenc}
\usepackage[T1]{fontenc}
\usepackage{geometry}
\geometry{a4paper}

\title{Book of Myre}
\author{Martin Evans\\martin@placeholder-software.co.uk}
\date{}

\begin{document}
\maketitle

\tableofcontents
\pagebreak[4]

\section{Introduction}
Myre is a software framework for C\#, it is designed to make building games in XNA quick and easy. Myre is split into several parts, each part provides some distinct functionality and you may pick and choose the different parts of Myre to include in your game as you wish.

\subsection{Parts of Myre}
\subsubsection{Myre}
The core of the Myre framework which all other parts depend upon, it must be referenced in your project. Myre contains useful Collections and Extension methods that are used throughout the rest of the framework.
\subsubsection{Myre.StateManagement}
A simple system for managing states within your game and the transition between them. The intended use for this is screens in a game, and handling slide-on and slide-off transitions as the user navigates through the menus.
\subsubsection{Myre.Entities}
Provides a system for managing the state of a game during gameplay. Gamestate is stored in entities, which are collections composed of "Properties" (data) and "Behaviours" (actions taken on the data). Myre.Entities is intended to be the core of the architecture of a game.
\subsubsection{Myre.Graphics.UI}
\subsubsection{Myre.Debugging}
\subsubsection{Myre.Debugging.UI}
\subsubsection{Myre.Graphics}
\subsubsection{Myre.Graphics.Content}
\subsubsection{Myre.Graphics.Pipeline}
\subsubsection{Myre.Physics}
\subsubsection{Myre.Serialisation}

\end{document}