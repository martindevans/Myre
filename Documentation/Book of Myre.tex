\documentclass{article}

\usepackage[utf8]{inputenc}
\usepackage[T1]{fontenc}
\usepackage{geometry}
\geometry{a4paper}

\title{Book of Myre}
\author{Martin Evans\\martin@placeholder-software.co.uk}
\date{}

\begin{document}
\maketitle

\tableofcontents
\pagebreak[4]

\section{How to read this book}
\subsection{New User}
If you are a new user, it is best to read the book from start to end as it is presented. The first section will introduce the different parts of Myre and their capabilities, going into very little technical detail - this will help you decide if Myre is for you.
\subsection{Current User}
If you are a current user, who has decided to use Myre or already use it then you should probably skip the introduction section. References for classes and examples are included in the next section, and these will help you quickly solve problems with Myre as you encounter them in your project.
\subsection{Myre Developer}
If you want to commit code to the Myre project that's great! The end of the book includes detailed descriptions of the internal workings of Myre, and rationalisations for the design decisions taken.
\pagebreak[4]

\section{Introduction}
Myre is a software framework for C\#, it is designed to make building games in XNA quick and easy. Myre is split into several parts, each part provides some distinct functionality and you may pick and choose the different parts of Myre to include in your game as you wish.
\subsection{Myre}
\subsection{Myre.StateManagement}
\subsection{Myre.Entities}
\subsection{Myre.UI}
\subsection{Myre.Debugging}
\subsection{Myre.Debugging.UI}
\subsection{Myre.Graphics}
\subsection{Myre.Graphics.Content}
\subsection{Myre.Graphics.Pipeline}
\subsection{Myre.Physics}
\subsection{Myre.Serialisation}

\section{User Reference}
\subsection{Myre}
\subsection{Myre.StateManagement}
\subsection{Myre.Entities}
\subsection{Myre.UI}
\subsection{Myre.Debugging}
\subsection{Myre.Debugging.UI}
\subsection{Myre.Graphics}
\subsection{Myre.Graphics.Content}
\subsection{Myre.Graphics.Pipeline}
\subsection{Myre.Physics}
\subsection{Myre.Serialisation}

\section{User Tutorials}

\section{Developer Reference}
\subsection{Myre}
\subsection{Myre.StateManagement}
\subsection{Myre.Entities}
\subsection{Myre.UI}
\subsection{Myre.Debugging}
\subsection{Myre.Debugging.UI}
\subsection{Myre.Graphics}
\subsection{Myre.Graphics.Content}
\subsection{Myre.Graphics.Pipeline}
\subsection{Myre.Physics}
\subsection{Myre.Serialisation}

\end{document}